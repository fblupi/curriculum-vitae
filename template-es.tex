%% Inicio del archivo `template-es.tex'.
%% Copyright 2006-2013 Xavier Danaux (xdanaux@gmail.com).
%
% This work may be distributed and/or modified under the
% conditions of the LaTeX Project Public License version 1.3c,
% available at http://www.latex-project.org/lppl/.

\documentclass[11pt,a4paper,sans]{moderncv} % font ('10pt', '11pt' or '12pt'), size ('a4paper', 'letterpaper', 'a5paper', 'legalpaper', 'executivepaper' y 'landscape') and font family ('sans' or 'roman')
\moderncvstyle{casual} % 'casual', 'classic', 'oldstyle' or 'banking'
\moderncvcolor{blue} % 'blue', 'orange', 'green', 'red', 'purple', 'grey' or 'black'
\renewcommand{\familydefault}{\sfdefault} % '\sfdefault' or '\rmdefault'
\nopagenumbers{}

\usepackage[utf8]{inputenc}

\usepackage[scale=0.75]{geometry}
\renewcommand*{\namefont}{\fontsize{42}{42}\mdseries\upshape}

\name{Francisco Javier}{Bolívar Lupiáñez}
\title{Ingeniero Informático}
\address{Carretera de Málaga, 119, 4ºC}{18015, Granada (España)}
\phone[mobile]{+34~601~189~876}
\email{francisco.bolivar.lupianez@gmail.com}
\social[github]{fblupi}
\social[linkedin]{fblupi}
\social[twitter]{fblupi}
\extrainfo{12 Septiembre 1994}
\photo[64pt][0.4pt]{photo.jpeg}
\usepackage[resetlabels]{multibib}
%----------------------------------------------------------------------------------
%            contenido
%----------------------------------------------------------------------------------
\begin{document}
\maketitle

\section{Formación académica}
\cventry{2016--2018}{Máster en Ingeniería Informática}{Universidad de Granada}{Granada}{\textit{9,364}}{Premio al mejor expediente académico de la promoción} 
\cventry{2012--2016}{Grado en Ingeniería Informática}{Universidad de Granada}{Granada}{\textit{8,378}}{Especialidad en Ingeniería del Software. Matrícula de honor en 9 asignaturas, incluyendo el Trabajo de Fin de Grado.}
\cventry{2010--2012}{Bachillerato}{Juan XXIII - Chana}{Granada}{\textit{9,32}}{Ciencias y tecnología. Premio al mejor expediente académico de la promoción.}

\section{Experiencia}
\cventry{2017--}{Desarrollador Full-Stack}{Nazaríes IT}{Granada}{}{Empecé como un junior, después de una etapa exitosa durante mis prácticas de empresa, hasta ser considerado senior y una de las piezas más importantes del equipo. Desarrollador en diversos proyectos con clientes tanto nacionales como internacionales. Aunque el proyecto principal en el que he trabajado ha sido en un entorno de IoT con un \textit{stack} tecnológico basado en Ruby on Rails, he trabajado entre medias también con otros proyectos en Ruby on Rails y otras tecnologías muy diversas: como una aplicación de escritorio con C++ para controlar drones, un \textit{back-end} puro con Spring, varias aplicaciones móviles híbridas usando Ionic, una plataforma de \textit{e-commerce} a medida con PHP o una importante plataforma de telemedicina también con PHP.}
\cventry{2017}{Estudiante de Prácticas}{Nazaríes IT}{Granada}{}{Prácticas de empresa correspondientes al máster en Ingeniería Informática en Nazaríes IT. Empecé a aprender y trabajar con Ruby on Rails.}
\cventry{2016--2017}{Becario de colaboración}{Departamento de Lenguajes y Sistemas Informáticos}{Granada}{}{Trabajando tanto en un proyecto de investigación encaminado al estudio y desarrollo de algoritmos para la visualización y documentación de esculturas de madera así como en el proyecto Atalaya3D procesando modelos de figuras 3D para su publicación en la web.}

\section{Idiomas}
\cvitemwithcomment{Español}{Competencia bilingüe o nativa}{}
\cvitemwithcomment{Inglés}{Competencia básica profesional}{B1 English Certificate. University of Cambridge}

\clearpage

\section{Conocimientos}
\cvitem{Back-end web development}{Ruby on Rails, Django, PHP, MySQL, PostgreSQL, MongoDB, Sidekiq}
\cvitem{Front-end web development}{Angular, Ionic, jQuery, HTML, CSS}
\cvitem{DevOps}{Docker, AWS}
\cvitem{Informática gráfica}{OpenGL, Blender, VTK, ITK, OpenCV}
\cvitem{Inteligencia artificial}{Redes neuronales, algoritmos evolutivos}
\cvitem{Otros}{Git, Scrum, UML, CUDA,  Qt, Photoshop}

\section{Publicaciones}
\renewcommand{\refname}{Artículos}
\nocite{*}
\bibliographystyle{plain}
\bibliography{publications}

\section{Otra información}
\cvlistitem{Permiso de circulación tipo B}

\end{document}

