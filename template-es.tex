%% Inicio del archivo `template-es.tex'.
%% Copyright 2006-2013 Xavier Danaux (xdanaux@gmail.com).
%
% This work may be distributed and/or modified under the
% conditions of the LaTeX Project Public License version 1.3c,
% available at http://www.latex-project.org/lppl/.

\documentclass[11pt,a4paper,sans]{moderncv}  % opciones posibles incluyen tamaño de fuente ('10pt', '11pt' and '12pt'), tamaño de papel ('a4paper', 'letterpaper', 'a5paper', 'legalpaper', 'executivepaper' y 'landscape') y familia de fuentes ('sans' y 'roman')

% temas de moderncv
\moderncvstyle{casual}                       % las opciones de estilo son 'casual' (por omision),'classic', 'oldstyle' y 'banking'
\moderncvcolor{blue}                         % opciones de color 'blue', 'orange', 'green', 'red', 'purple', 'grey' y 'black'
\renewcommand{\familydefault}{\sfdefault}    % para seleccionar la fuente por omision, use '\sfdefault' para la fuente sans serif, '\rmdefault' para la fuente roman, o cualquier nombre de fuente
\nopagenumbers{}                             % elimine el comentario para suprimir la numeracion automatica de las paginas para CVs mayores a una pagina

% codificacion de caracteres
\usepackage[utf8]{inputenc}                  % reemplace con su codificacion
%\usepackage{CJKutf8}                        % si necesita usa CJK para redactar su CV en chino, japones o coreano

% ajustes para los margenes de pagina
\usepackage[scale=0.75]{geometry}
%\setlength{\hintscolumnwidth}{3cm}           % si desea cambiar el ancho de la columna para las fechas
\renewcommand*{\namefont}{\fontsize{42}{42}\mdseries\upshape}

% datos personales
\name{Francisco Javier}{Bolívar Lupiáñez}
\title{Graduado en Ingeniería Informática}
\address{Carretera de Málaga, 119, 4ºC}{18015, Granada (España)}
\phone[mobile]{+34~601~189~876}
\email{franciscojavierbolivarlupianez@gmail.com}
\homepage{http://fblupi.github.io/}
\social[github]{fblupi}
\social[linkedin]{fblupi}
\social[twitter]{fblupi}
\extrainfo{12 Septiembre 1994}
\photo[64pt][0.4pt]{photo}
%\quote{Alguna cita (opcional)}

% para mostrar etiquetas numericas en la bibliografia (por omision no se muestran etiquetas), solo es util si desea incluir citas en en CV
%\makeatletter
%\renewcommand*{\bibliographyitemlabel}{\@biblabel{\arabic{enumiv}}}
%\makeatother

% bibliografia con varias fuentes
\usepackage{multibib}
\newcites{book,misc}{{Libros},{Artículos}}
%----------------------------------------------------------------------------------
%            contenido
%----------------------------------------------------------------------------------
\begin{document}
%\begin{CJK*}{UTF8}{gbsn}                     % para redactar el CV en chino usando CJK
\maketitle

\section{Formación académica}
\cventry{2016--}{Máster en Ingeniería Informática}{Universidad de Granada}{Granada}{}{} 
\cventry{2012--2016}{Grado en Ingeniería Informática}{Universidad de Granada}{Granada}{\textit{8,378}}{Especialidad en Ingeniería del Software. Matrícula de honor en 9 asignaturas, incluyendo el Trabajo de Fin de Grado.}
\cventry{2010--2012}{Bachillerato}{Juan XXIII - Chana}{Granada}{\textit{9,32}}{Ciencias y tecnología. Matrícula de honor con mejor nota media de la promoción.}

\section{Experiencia}
\cventry{2016--}{Becario de colaboración}{Departamento de Lenguajes y Sistemas Informáticos}{Granada}{}{Trabajando tanto en un proyecto de investigación encaminado al estudio y desarrollo de algoritmos para la visualización y documentación de esculturas de madera así como en el proyecto Atalaya3D procesando modelos de figuras 3D para su publicación en la web.}

\section{Idiomas}
\cvitemwithcomment{Español}{Competencia bilingüe o nativa}{}
\cvitemwithcomment{Inglés}{Competencia básica profesional}{B1 English Certificate. University of Cambridge}

%\section{Conocimientos}
%\cvdoubleitem{Aplicaciones web}{Symfony, Flask, Django, Bootstrap}{Bases de datos}{MongoDB, MySQL, OracleSQL}
%\cvdoubleitem{Dispositivos móviles}{Android}{Informática gráfica}{OpenGL, WebGL, Java3D, Three.js, X3D, VTK, ITK}
%\cvdoubleitem{Cloud Computing}{Ansible, Vagrant, Docker, AWS}{Otros}{}

\section{Conocimientos}
\cvitem{Desarrollo web}{Symfony, Flask, Django, Bootstrap}
\cvitem{Bases de datos}{MongoDB, MySQL, OracleSQL}
\cvitem{Dispositivos móviles}{Android}
\cvitem{Informática gráfica}{OpenGL, WebGL, Blender, Java3D, Three.js, X3D, VTK, ITK}
\cvitem{Cloud Computing}{Ansible, Vagrant, Docker, AWS}
\cvitem{Inteligencia artificial}{Redes neuronales, algoritmos evolutivos}
\cvitem{Otros}{Git, Scrum, UML, CUDA,  Qt}

%\section{Extra 1}
%\cvlistitem{Tema 1}
%\cvlistitem{Tema 2}
%\cvlistitem{Tema 3}

%\renewcommand{\listitemsymbol}{-~}            % para cambiar el simbolo para las listas

%\section{Extra 2}
%\cvlistdoubleitem{Tema 1}{Tema 4}
%\cvlistdoubleitem{Tema 2}{Tema 5\cite{book1}}
%\cvlistdoubleitem{Tema 3}{}

% Las publicaciones tomadas de un archivo de BibTeX sin usar multibib\renewcommand*{\bibliographyitemlabel}{\@biblabel{\arabic{enumiv}}}

\nocite{*}
%\bibliographystyle{plain}
%\bibliography{publications}                   % 'publications' es el nombre del archivo BibTeX

% Las publicaciones tomadas de un archivo BibTeX usando el paquete multibib
\section{Publicaciones}
%\nocitebook{book1,book2}
%\bibliographystylebook{plain}
%\bibliographybook{publications}              % 'publications' es el nombre del archivo BibTeX
\nocitemisc{ceig.20161311}
\bibliographystylemisc{plain}
\bibliographymisc{publications}              % 'publications' es el nombre del archivo BibTeX

\section{Otra información}
\cvlistitem{Permiso de circulación tipo B}

%\clearpage\end{CJK*}                          % si esta redactando su CV en chino usando CJK, \clearpage es requerido por fancyhdr para que funcione correctamente con CJK, aunque esto eliminara la numeracion de pagina al dejar \lastpage como no definido
\end{document}


%% fin del archivo `template-es.tex'.
