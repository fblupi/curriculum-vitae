%% Inicio del archivo `template-es.tex'.
%% Copyright 2006-2013 Xavier Danaux (xdanaux@gmail.com).
%
% This work may be distributed and/or modified under the
% conditions of the LaTeX Project Public License version 1.3c,
% available at http://www.latex-project.org/lppl/.

\documentclass[11pt,a4paper,sans]{moderncv}  % opciones posibles incluyen tamaño de fuente ('10pt', '11pt' and '12pt'), tamaño de papel ('a4paper', 'letterpaper', 'a5paper', 'legalpaper', 'executivepaper' y 'landscape') y familia de fuentes ('sans' y 'roman')

% temas de moderncv
\moderncvstyle{casual}                       % las opciones de estilo son 'casual' (por omision),'classic', 'oldstyle' y 'banking'
\moderncvcolor{blue}                         % opciones de color 'blue', 'orange', 'green', 'red', 'purple', 'grey' y 'black'
\renewcommand{\familydefault}{\sfdefault}    % para seleccionar la fuente por omision, use '\sfdefault' para la fuente sans serif, '\rmdefault' para la fuente roman, o cualquier nombre de fuente
\nopagenumbers{}                             % elimine el comentario para suprimir la numeracion automatica de las paginas para CVs mayores a una pagina

% codificacion de caracteres
\usepackage[utf8]{inputenc}                  % reemplace con su codificacion
%\usepackage{CJKutf8}                        % si necesita usa CJK para redactar su CV en chino, japones o coreano

% ajustes para los margenes de pagina
\usepackage[scale=0.75]{geometry}
%\setlength{\hintscolumnwidth}{3cm}           % si desea cambiar el ancho de la columna para las fechas
\renewcommand*{\namefont}{\fontsize{42}{42}\mdseries\upshape}

% datos personales
\name{Francisco Javier}{Bolívar Lupiáñez}
\title{M.Sc. in Computer Science}
\address{Carretera de Málaga, 119, 4ºC}{18015, Granada (Spain)}
\phone[mobile]{+34~601~189~876}
\email{francisco.bolivar.lupianez@gmail.com}
\homepage{fblupi.es}
\social[github]{fblupi}
\social[linkedin]{fblupi}
\social[twitter]{fblupi}
\extrainfo{12 September 1994}
\photo[64pt][0.4pt]{photo.jpeg}
%\quote{Alguna cita (opcional)}

% para mostrar etiquetas numericas en la bibliografia (por omision no se muestran etiquetas), solo es util si desea incluir citas en en CV
%\makeatletter
%\renewcommand*{\bibliographyitemlabel}{\@biblabel{\arabic{enumiv}}}
%\makeatother

% bibliografia con varias fuentes
\usepackage{multibib}
%----------------------------------------------------------------------------------
%            contenido
%----------------------------------------------------------------------------------
\begin{document}
%\begin{CJK*}{UTF8}{gbsn}                     % para redactar el CV en chino usando CJK
\maketitle

\section{Education}
\cventry{2016--2018}{M.Sc. in Computer Science}{Universidad de Granada}{Granada}{\textit{9.364}}{Best academic results award.} 
\cventry{2012--2016}{B.Sc. in Computer Science}{Universidad de Granada}{Granada}{\textit{8.378}}{Special mention in Software Engineering. I was graded with honours in 9 courses, including the End-of-Degree Project.}
\cventry{2010--2012}{High School}{Juan XXIII - Chana}{Granada}{\textit{9.32}}{Science and Technology. Best academic results award.}

\section{Experience}
\cventry{2017--}{Developer}{Nazaríes IT}{Granada}{}{Developer in several projects with both national and international clients. Working with diverse technologies but mainly with Ruby on Rails.}
\cventry{2017}{Internship Student}{Nazaríes IT}{Granada}{}{Internship in Nazaríes IT corresponding to M.Sc. in Computer Science.}
\cventry{2016-2017}{Scholarship Assistant}{Department of Lenguajes y Sistemas Informáticos}{Granada}{}{
Working in an investigation project, for studying and devoloping algorithms to view and document wood sculptures, and in Atalaya3D project, processing 3D models to be uploaded to the web.}

\section{Languages}
\cvitemwithcomment{Spanish}{Native proficiency}{}
\cvitemwithcomment{English}{Professional working proficiency}{B1 English Certificate. University of Cambridge}

%\section{Conocimientos}
%\cvdoubleitem{Aplicaciones web}{Symfony, Flask, Django, Bootstrap}{Bases de datos}{MongoDB, MySQL, OracleSQL}
%\cvdoubleitem{Dispositivos móviles}{Android}{Informática gráfica}{OpenGL, WebGL, Java3D, Three.js, X3D, VTK, ITK}
%\cvdoubleitem{Cloud Computing}{Ansible, Vagrant, Docker, AWS}{Otros}{}

\section{Knowledge}
\cvitem{Web development}{Ruby on Rails, Django, Bootstrap}
\cvitem{Database}{MongoDB, MySQL}
\cvitem{Mobile applications}{Android, Ionic (Angular)}
\cvitem{Computer Graphics}{OpenGL, WebGL, Blender, Java3D, Three.js, X3D, VTK, ITK, OpenCV}
\cvitem{Cloud Computing}{Ansible, Vagrant, Docker, AWS}
\cvitem{Artificial intelligence}{Neural Networks, Evolutionary Algorithms}
\cvitem{Others}{Git, Scrum, UML, CUDA,  Qt}

%\section{Extra 1}
%\cvlistitem{Tema 1}
%\cvlistitem{Tema 2}
%\cvlistitem{Tema 3}

%\renewcommand{\listitemsymbol}{-~}            % para cambiar el simbolo para las listas

%\section{Extra 2}
%\cvlistdoubleitem{Tema 1}{Tema 4}
%\cvlistdoubleitem{Tema 2}{Tema 5\cite{book1}}
%\cvlistdoubleitem{Tema 3}{}

% Las publicaciones tomadas de un archivo de BibTeX sin usar multibib\renewcommand*{\bibliographyitemlabel}{\@biblabel{\arabic{enumiv}}}

%\nocite{*}
%\bibliographystyle{plain}
%\bibliography{publications}                   % 'publications' es el nombre del archivo BibTeX

% Las publicaciones tomadas de un archivo BibTeX usando el paquete multibib
\section{Publications}
\renewcommand{\refname}{Papers}
\nocite{*}
\bibliographystyle{plain}
\bibliography{publications}              % 'publications' es el nombre del archivo BibTeX

\section{Miscellaneous}
\cvlistitem{Driving License B}

%\clearpage\end{CJK*}                          % si esta redactando su CV en chino usando CJK, \clearpage es requerido por fancyhdr para que funcione correctamente con CJK, aunque esto eliminara la numeracion de pagina al dejar \lastpage como no definido
\end{document}


%% fin del archivo `template-es.tex'.
